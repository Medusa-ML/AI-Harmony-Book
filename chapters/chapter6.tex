\setchapterpreamble[u]{\margintoc}
\chapter{Self-Driving with Statistics}
\labch{driving}

\textit{"We think of automation as a machine doing a task that a human used to do... you might think that means a human does nothing. But in fact there's abundant literature that shows the human is not incurring no workload, the human is now doing a different task and that task tends to be monitoring, a vigilance task, looking for rare events...that is a task that humans are not well-equipped to do." - Dr. Michael Nees, 2021} \cite{nees2021}


\section{The Dangers of Semi-Autonomy}

Two hundred years ago, horses were the most sophisticated mode of transportation. They possessed an innate ability to navigate terrain and avoid obstacles, even without a human rider at the reins. People often took comfort in the horse's natural instincts, which allowed for moments of respite during long journeys. Fast forward to today, where we have vehicles that claim to possess similar levels of autonomy, but with significantly more horsepower (pun intended).

One might be tempted to compare these self-driving cars to our trusty equine friends, imagining a world where vehicles, like horses, can be left to their own devices. Alas, this comparison is a misleading one, as it creates the illusion that our self-driving cars are more capable than they currently are.

The National Highway Traffic Safety Administration (NHTSA) has devised a six-level classification system to describe vehicle autonomy, ranging from Level 0 (no automation) to Level 5 (full automation). Most commercially available vehicles today hover between Levels 2 and 3, providing advanced driver assistance but requiring constant human oversight. This semi-autonomous state can lull drivers into a false sense of security, prompting them to disengage from the driving task in a manner that might have been acceptable during the days of horse-drawn carriages but is decidedly dangerous in the modern era.

In his book "Robot Take the Wheel," author Jason Torchinsky offers a compelling argument against semi-autonomy, dubbing it "stupid" in a section bearing the same name. Torchinsky highlights the impracticality of expecting a driver who has relinquished control to a semi-autonomous system to suddenly take over in a moment of crisis. Manufacturer warnings, while intended to encourage drivers to stay attentive, often go unheeded, creating a precarious situation where those behind the wheel are ill-prepared to intervene when the technology falters.\cite{torchinskyboeckmann2019}

The world of transportation extends beyond the realm of four-wheeled automobiles. The skies above host a veritable ballet of commercial aircraft, which rely on sophisticated autopilot systems to ferry passengers and cargo across vast distances. These systems differ in important ways from automobile autopilot and have significantly different infrastructure (not to mention terrain, or lack thereof).In aviation, autopilots demand constant monitoring and communication between the pilot, the aircraft, and air traffic control. In many ways, the intricacies and collaboration required for safe air travel can serve as a model for understanding the complexities of developing and implementing truly autonomous ground vehicles.

\section{Comparing Autopilot Systems}

Airplanes vs. Teslas: Similarities and Differences

Communication and Coordination with External Systems

Monitoring and Human Intervention in Autopilot Systems

\section{Who Should The Car Kill?}

The Trolley Problem and Autonomous Decision-Making

Outsourcing Responsibility: Legal and Moral Considerations

\section{Driving Infrastruture}

Maps
-Concept Drift: The Challenge of Ever-Changing Environments
-Pileup Crash \cite{pileupcrash}

Roads
-Urban Design: Retrofitting Cities for Self-Driving Cars

Sensors
-Dirty Sensors \sidenote{You can't get your self-driving car dirty either, it'll mess up those sensors.}

Software 
-(OpenPilot \cite{openpilot})
-Inspection and bankruptcy?
-OTA Crash \cite{otacrash}

Communications
-Other Cars
-People, Animals etc.. The Impact of New Technologies and Trends on Models


\section{See You In Court!}

Multicollinearity and Mathematical Chaos (reprise)

Explainability and Accountability in Court

Trustworthy ML O'Reilly \cite{trustworthyml}

The Case for Inherently Interpretable Models

\section{Alternative Approaches}

It's a train!

From Self-Driving Forklifts to Moving Shelves

The Role of Vehicle-to-Vehicle Communication

Remote Control and Supervised Autonomy

The Shift in Transportation Preferences

The Balance Between Autonomy and Regulation
