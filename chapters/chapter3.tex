\setchapterpreamble[u]{\margintoc}
\chapter{Creativity and Decision Making with AI}
\labch{intro}

\textit{“I hope for some sort of peace—but I fear that machines are ahead of morals by some centuries and when morals catch up there'll be no reason for any of it.”} Harry Truman, 1945 \cite{McCullough1992}

%TODO discuss these books
%\cite{Oneil2017}
%\cite{Perez2019}
%\cite{Blackman2022Jul}
%\cite{Christian2020}

\section{Theories of Creativity}

Is creativity combining exsiting things, or is it coming up with something new?

Does progress slow down becasue we heavily rely (more than usual) on the work of the past to generate future work?

We always rely on the past to create future work... this is not new

The prompt you gave to ChatGPT is the creative act now. Just like you can do boring things with a paint roller, or do somethig creative.

Input and output... the AI is just a machine, a static but complicated one. Small changes in inputs will make wildly new outputs.

\section{Getting creative with a Mikita Drill}

Modern Artificial Intelligence and a Mikita drill are two complex but ultimately deterministic systems. Though they are both complex and require a user to understand how to use them, the user input ultimately decides the output.

Modern AI is a complex system of algorithms, data, and analytics that can be used to solve complex problems. AI systems can learn from data, identify patterns, and make predictions about the future. AI systems are typically used to automate and assist human decision making. AI systems are programmed with specific objectives and goals, and the user input decides the output. For example, an AI system could be programmed to solve a mathematical problem and the user input would determine the parameters of the problem and the output would be the solution.

A Mikita drill is also a complex deterministic system. The user input is limited to the type of drill bit, the speed of the drill, and the direction of the drill. The output is determined by these inputs, as the drill will only drill in the direction and speed determined by the user. The user also has to choose the correct drill bit to ensure the drill can do the job correctly and safely.

Overall, both modern AI and a Mikita drill are complex but ultimately deterministic systems. The user input, however limited it may be, ultimately controls the output of the system. Both systems require a user to understand how to use them and to make the correct input to get the desired output.

Creative ways to use a drill.
\begin{itemize}
\item What you point it at
\item The bits you use
\item The pace and speed of your finger on the trigger
\item The amount of pressure you put on the drill and the wall
\end{itemize}


\section{Garbage In, Garbage Out}

You are essentially programming with data, so if your data sucks so will your prediction, you also really can't generalize, only correllate.

\section{Garbage In, New Perspective Out?}

What about cross-domain models, where maybe I train with a poetry dataset, and point my language model on nonfiction.... hmmm.

\section{Concept Drift}

When the meaning of truth changes.

\section{The Impossibility of Fairness}

Representation, "fixing the training set" \sidecite{Christian2020}, or the Impossibility of Fairness from a model.

\section{What Are We Prediciting Again?}

are you predicting the right thing? Are you really predicting how valuable the company is or just whether it'll be the next meme stock?

\section{Humans Love Computers}

Working together seems like a good idea, but how. Talk about 2 percent model control and model uptake. 

Also talk about Kasparov's tournament and Thinking for yourself in the age of AI. \sidecite{mansharamani2020}

\section{Key Takeaways}

\begin{itemize}
    \item \textbf{Deep learning models are fundamentally large unscientific regressions} they are trained to create a function that maps input data to output data.
    \item \textbf{Deep learning models are chaotic systems containing millions of interacting parameters} they are not designed to be explained or created in a way that their weights can be used for scientific analysis. They find reasonable answers and don't care how they get there. Multicolinearity (understanding the relationship of an input and output) and feature importance (understanding which inputs are most important) are only understandable with a high level of statistical error.
    \item \textbf{Small changes in inputs of a deep learning model may dramatically change the outputs} deep learning models are complex deterministic systems that can exhibit chaotic behavior. Their inner workings are functionally unknowable and practically impossible to test.
    \item \textbf{Deep learning models can have impressive and useful outputs, but the creators of models should be encouraged to highlight their failures and limitations.} Machine learning engineers might be more keen to highlight failures and limitations if they are encouraged to do so by their users, managers and investors.
\end{itemize}

