\setchapterpreamble[u]{\margintoc}
\chapter{Revolutionary for Whom?}
\labch{rev}

\textit{"The inhabitant of London could order by telephone, sipping his morning tea in bed, the various products of the whole earth -- he could at the same time and by the same means adventure his wealth in the natural resources and new enterprise of any quarter of the world -- he could secure forthwith, if he wished, cheap and comfortable means of transit to any country or climate without passport or other formality."} John Maynard Keynes, 1920 \cite{Keynes2012}




\section{The AI Advantage in the Workplace}

Challenge the notion of Superior Human Intelligence

Compare the use of AI tools to the use of a calculator in math. Let computers do what they are good at!

Talk about embracing AI in Daily Tasks and Careers

\section{The Limitations of AI-generated Content}

\textit{"Another definition of modernity: conversations can be more and more completely reconstructed with clips from other conversations taking place at the same time on the planet.", "You are alive in inverse proportion to the density of cliches in your writing."}\cite{procrustes} 

AI-generated Conversations are not Conversations...

ChatGPT as a "Blurry JPEG of the Web" explain what this article is trying to explain \cite{newyorkerChatGPTBlurry}

Comparing AI to Traditional Search Engines, one looks up information, the other generates it from as sophisticated regression, and it might not always be right. 

\textit{"(Traditional) search engines are databases, organized collections of data that can be stored, updated, and retrieved at will. (Traditional) search engines are indexes. a form of database, that connect things like keywords to URLs; they can be swiftly updated, incrementally, bit by bit (as when you update a phone number in the database that holds your contacts).}

\textit{Large language models do something very different: they are not databases; they are text predictors, turbocharged versions of autocomplete. Fundamentally, what they learn are relationships between bits of text, like words, phrases, even whole sentences. And they use those relationships to predict other bits of text. And then they do something almost magical: they paraphrase those bits of texts, almost like a thesaurus but much much better. But as they do so, as they glom stuff together, something often gets lost in translation: which bits of text do and do not truly belong together."} Gary Marcus, 2023 \cite{marcus2023}

\section{The "Organic Content" Market}

The Role of AI in Content Production

\section{A Framework for Navigating AI's Impact on Careers}

\section{Worse, but still on top}

Krohn Monkeys and AI Superiority \cite{KrohnTED}

\section{Dead Inside}

\textit{"If you know, in the morning, what your day looks like with any precision you are a little bit dead - the more precision the more dead you are."}\cite{procrustes}

\section{Take it, or Be Left Behind}

point by point refutaiton of this idiotic letter. \cite{dumbestletter}

Should we let machines flood our information channels with propaganda and untruth? Just stop getting your news from social media, subscribe to the New York Times or WSJ and call it a day, this is already a problem with state actors flooding the zone with propaganda.

Should we automate away all the jobs, including the fulfilling ones? If you find automatable work fulfilling, start an Etsy shop and tell everyone you did everything by hand.... some people even sharpen pencils by hand, but stop pretending you can regulate this away. 

Should we develop nonhuman minds that might eventually outnumber, outsmart, obsolete and replace us? This is silly, Can they really outsmart us? Can they really replace us? Computers have been better at many tasks than us for forever, just because they can score better than 90 percent of humanity on the SAT does not mean that they can replace us. 

Should we risk loss of control of our civilization? WTF does this mean? Who controls it now?

