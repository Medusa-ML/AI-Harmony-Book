%%%%%%%%%%%%%%%%%%%%%%%%%%%%%%%%%%%%%%%%%
% kaobook
% LaTeX Class
% Version 0.9.8 (2021/08/23)
%
% This template originates from:
% https://www.LaTeXTemplates.com
%
% For the latest template development version and to make contributions:
% https://github.com/fmarotta/kaobook
%
% Authors:
% Federico Marotta (federicomarotta@mail.com)
% Based on the doctoral thesis of Ken Arroyo Ohori (https://3d.bk.tudelft.nl/ken/en)
% and on the Tufte-LaTeX class.
% Modified for LaTeX Templates by Vel (vel@latextemplates.com)
%
% License:
% LPPL (see included MANIFEST.md file)
%
%%%%%%%%%%%%%%%%%%%%%%%%%%%%%%%%%%%%%%%%%

%----------------------------------------------------------------------------------------
%	EXAMPLE AND DOCUMENTATION OF THE KAOBOOK CLASS
%----------------------------------------------------------------------------------------

\documentclass[
    letterpaper, % Page size
    fontsize=10pt, % Base font size
    twoside=false, % Use different layouts for even and odd pages (in particular, if twoside=true, the margin column will be always on the outside)
	%open=any, % If twoside=true, uncomment this to force new chapters to start on any page, not only on right (odd) pages
	%chapterentrydots=true, % Uncomment to output dots from the chapter name to the page number in the table of contents
	numbers=noenddot, % Comment to output dots after chapter numbers; the most common values for this option are: enddot, noenddot and auto (see the KOMAScript documentation for an in-depth explanation)
]{kaobook}

%----------------------------------------------------------------------------------------
%	PACKAGES AND OTHER DOCUMENT CONFIGURATIONS
%----------------------------------------------------------------------------------------

% Choose the language
\ifxetexorluatex
	\usepackage{polyglossia}
	\setmainlanguage{english}
\else
	\usepackage[english]{babel} % Load characters and hyphenation
\fi
\usepackage[english=british]{csquotes}	% English quotes

% Load packages for testing
\usepackage{blindtext} % Print text without any meaning for testing purposes
%\usepackage{showframe} % Uncomment to show boxes around the text area, margin, header and footer
%\usepackage{showlabels} % Uncomment to output the content of \label commands to the document where they are used

% Load the bibliography package
\usepackage{kaobiblio}
\addbibresource{main.bib} % Bibliography file

% Load mathematical packages for theorems and related environments
\usepackage[framed=true]{kaotheorems}

% Load the package for hyperreferences
\usepackage{kaorefs}

\graphicspath{{examples/documentation/images/}{images/}} % Paths in which to look for images

\makeindex[columns=3, title=Alphabetical Index, intoc] % Make LaTeX produce the files required to compile the index

\makeglossaries % Make LaTeX produce the files required to compile the glossary
\newglossaryentry{computer}{
	name=computer,
	description={is a programmable machine that receives input, stores and manipulates data, and provides output in a useful format}
}

% Glossary entries (used in text with e.g. \acrfull{fpsLabel} or \acrshort{fpsLabel})
\newacronym[longplural={Frames per Second}]{fpsLabel}{FPS}{Frame per Second}
\newacronym[longplural={Tables of Contents}]{tocLabel}{TOC}{Table of Contents}

 % Include the glossary definitions

\makenomenclature % Make LaTeX produce the files required to compile the nomenclature

% Reset sidenote counter at chapters
%\counterwithin*{sidenote}{chapter}

%----------------------------------------------------------------------------------------

\begin{document}

%----------------------------------------------------------------------------------------
%	BOOK INFORMATION
%----------------------------------------------------------------------------------------


\title[Untitled Project]{Untitled \\ Project}
\subtitle{Kicking the tires of ML models and stuff...}

\author[Brad Flaugher]{Brad Flaugher}

\date{\today}

\publishers{}

%----------------------------------------------------------------------------------------

\frontmatter % Denotes the start of the pre-document content, uses roman numerals

%----------------------------------------------------------------------------------------
%	OPENING PAGE
%----------------------------------------------------------------------------------------

%\makeatletter
%\extratitle{
%	% In the title page, the title is vspaced by 9.5\baselineskip
%	\vspace*{9\baselineskip}
%	\vspace*{\parskip}
%	\begin{center}
%		% In the title page, \huge is set after the komafont for title
%		\usekomafont{title}\huge\@title
%	\end{center}
%}
%\makeatother

%----------------------------------------------------------------------------------------
%	COPYRIGHT PAGE
%----------------------------------------------------------------------------------------

\makeatletter
\uppertitleback{\@titlehead} % Header

\lowertitleback{
	\textbf{Disclaimer}\\
	You can edit this page to suit your needs. For instance, here we have a no copyright statement, a colophon and some other information. This page is based on the corresponding page of Ken Arroyo Ohori's thesis, with minimal changes.
	
	\medskip
	
	\textbf{No copyright}\\
	\cczero\ This book is released into the public domain using the CC0 code. To the extent possible under law, I waive all copyright and related or neighbouring rights to this work.
	
	To view a copy of the CC0 code, visit: \\\url{http://creativecommons.org/publicdomain/zero/1.0/}
	
	\medskip
	
	\textbf{Colophon} \\
	This document was typeset with the help of \href{https://sourceforge.net/projects/koma-script/}{\KOMAScript} and \href{https://www.latex-project.org/}{\LaTeX} using the \href{https://github.com/fmarotta/kaobook/}{kaobook} class.
	
	The source code of this book is available at:\\\url{https://github.com/fmarotta/kaobook}
	
	(You are welcome to contribute!)
	
	\medskip
	
	\textbf{Publisher} \\
	First printed in May 2019 by \@publishers
}
\makeatother

%----------------------------------------------------------------------------------------
%	DEDICATION
%----------------------------------------------------------------------------------------

\dedication{
	The harmony of the world is made manifest in Form and Number, and the heart and soul and all the poetry of Natural Philosophy are embodied in the concept of mathematical beauty.\\
	\flushright -- D'Arcy Wentworth Thompson
}

%----------------------------------------------------------------------------------------
%	OUTPUT TITLE PAGE AND PREVIOUS
%----------------------------------------------------------------------------------------

% Note that \maketitle outputs the pages before here

\maketitle

%----------------------------------------------------------------------------------------
%	PREFACE
%----------------------------------------------------------------------------------------

\chapter*{Preface}
\addcontentsline{toc}{chapter}{Preface} % Add the preface to the table of contents as a chapter

This book is a work in progress, I hope it helps demystify the world of deep learning as I understand it.

Humans won't be able to control superintelligent AI, talk about that here\cite{Andreu2021}

Talk about Bostrom and GPAI here, and Erdi's answer to that. \cite{Erdi2019} \cite{Bostrom2014}

Talk about the alignment problem and Ethical freakouts about AI. Talk about the big 3 from 
\cite{Christian2020}
\cite{Blackman2022Jul}

Funding and startups, everybody is doing it, I'm trying to make sense of it


\begin{flushright}
	\textit{Brad Flaugher}
\end{flushright}

\index{preface}

%----------------------------------------------------------------------------------------
%	TABLE OF CONTENTS & LIST OF FIGURES/TABLES
%----------------------------------------------------------------------------------------

\begingroup % Local scope for the following commands

% Define the style for the TOC, LOF, and LOT
%\setstretch{1} % Uncomment to modify line spacing in the ToC
%\hypersetup{linkcolor=blue} % Uncomment to set the colour of links in the ToC
\setlength{\textheight}{230\hscale} % Manually adjust the height of the ToC pages

% Turn on compatibility mode for the etoc package
\etocstandarddisplaystyle % "toc display" as if etoc was not loaded
\etocstandardlines % "toc lines as if etoc was not loaded

\tableofcontents % Output the table of contents

\listoffigures % Output the list of figures

% Comment both of the following lines to have the LOF and the LOT on 
% different pages
\let\cleardoublepage\bigskip
\let\clearpage\bigskip

\listoftables % Output the list of tables

\listoflstlistings % Output the list of listings

\endgroup

%----------------------------------------------------------------------------------------
%	MAIN BODY
%----------------------------------------------------------------------------------------

\mainmatter % Denotes the start of the main document content, resets page numbering and uses arabic numbers
\setchapterstyle{kao} % Choose the default chapter heading style

\chapter*{Introduction}
\addcontentsline{toc}{chapter}{Introduction} % Add the intro to the table of contents as a chapter

Artificial Intelligence (AI) is not magic. It's here, and it's changing the world. Deep learning is one of the most exciting and popular fields of AI, but it's not the same as the good old-fashioned rules-based AI of the past. Deep learning involves training models by repeatedly showing them large datasets and allowing the models to infer the rules between input and output data.

Deep learning models are trained on data, almost like humans. However, the quality of the data is critical to the functioning of the model. For stable and well-understood environments like chess, chemistry or Newtonian physics, we can collect and generate data and deep learning can do a tremendous amount of useful work for us. In less stable environments where the rules of the day sometimes do not reflect the rules of the past, deep learning can be less helpful and even cause real harm when naively deployed. 

I'll discuss dozens of models in detail but consider any data collected that involves complex social interactions. Start with family interactions, then romantic ones and then consider that topics like advertising, trading stocks, credit scoring and even hate speech and threat detection all might have a dynamic social component to them. Models predictive power will suffer if the past does not look like the future. This is a problem for deep learning models that are trained on old data.

Also consider that as a creator of deep learning models, I can use a model to editorialize. I can train a model on data that fits my worldview instead of data that fits the world as it is. As a user or investor, how would you stop me? My stock trading model would be the first one to get me in trouble. If it was regularly monitored and managed, my model could do about as much harm as a troublesome employee could. 

But what about my model that is used to provide online dating advice, credit scores, or acceptance to students to elite universities? It might take a few years before my editorializing was found out, depending on how well it was managed and how egregious my model's outputs are.

Some of these problems are small potatoes, who cares if online dating sites don't do good science to suggest matches? What about full-self driving though? The realization of autonomous cars we're told is perpetually years away, but is it really possible with our current roads, laws and infrastructure? What happens if a model stops being updated, and it becomes the height of fashion for kids to wear shirts with street-signs on them? Do all of the cars stop working? Also, we're calling this autopilot, but don't pilots in planes need to file flight plans, speak to other planes and take instructions from a tower? Do our cars need to do that too? 

If you understand the type of AI that is being used (rules-based or deep learning), the data it was trained on, and monitor the AI in production you are on the road to success. But, what if I told you I could use the outputs from the model you just spent a billion dollars building and very easily create a new model that performs almost as well, and spend almost nothing building it. Furthermore, if you took me to court I could show my work and prove that I made the model myself. Would this change the way you invest in artificial intelligence? Would it change the way you develop and share your models?

Back to self-driving cars, Here is how that could pan out. Imagine Honda comes from behind and creates the worlds greatest fully self-driving car and it cost them only \$1 billion to make. Now imagine I buy a couple of those Hondas, maybe 100 of them for \$30,000 each, then pay 100 drivers \$100,000 per year to drive them, and equip each honda with \$70,000 of my own computers and sensors. At the end of this endeavour I could theoretically create a decent machine learning model using the "Honda's" data for 20\% of their investment. There are of course other costs I would incur, but if you come along with me and assume that the data is the most important component of the model, then I have all of the data I need on the cheap.

I'll disucss this in detail in chapter 5, but for now let me tell you I'm skeptical of full self-driving without a lot of infrastructure changes, but I'm not skeptical of the ability to create a model that can drive a car. I'm skeptical of the ability to create a model that can drive a car in a way that is safe, reliable and that will hold up in court.   

The cost of innovating is well known, and this is why patent protection exists. But deep learning models are trained on data that is free or easy to copy and in a way that produces slightly different and seemingly random inner-workings on each training run, if this is the case then I don't see many defensible positions for innovators in machine learning. Let's consider another hypothetical from pharmacology, imagine Pfizer invented the cancer curing pill, then I put that pill in a machine that came up with a significantly different formulation that achieved the same results, and when you gave it to the expert witness chemists they would have to say "the chemistry of these two pills is fundamentally different, but they both cure cancer", which would make it very hard to defend the original cancer pill's patent in court.

Because of the way they are trained, deep learning models introduce real mathematical chaos wherever they are deployed. This leads me to a few conclusions that I'll give you here in the introduction, but explain in detail in the meat of this book: 

\begin{enumerate}
\item Deep learning models can make statistically informed decisions based on the data they were previously shown, but because of their size they can produce seemingly random results. Poor data collection (or editorializing) leads to poor results. Anything using deep learning models cannot be used by itself to make critical decisions. Said differently, there must be supervision and outside control wherever deep learning is involved.
\item Any decision involving deep learning models is functionally unexplainable, and therefore likely to get someone in trouble in court. Any domain where deep learning is used to make a decision and then asked to explain in detail how that decision was made should be greeted with a shrug from the witness stand.
\item Deep learning intellectual property is indefensible, it is built in a way that is both easily copyable, and impossible to verify that it was actually copied.
\end{enumerate}

Deep learning introduces challenges for some, but opportunities for others. I am a member of the \href{https://www.fsf.org/}{Free Software Foundation}, and from my perspective deep learning models are one type of software that might inherently support the foundation's purpose. The purpose of the foundation is to promote the universal freedom to distribute and modify computer software without restriction. If deep learning models are simultaneously powerful and free, they become rocket fuel for innovation. 

This is the silver lining to the "myths" that I'd like to discuss in this book. Deep learning is messy, data science is hard, but as a tool deep learning is absolutely mindblowing. I can rank order my emails by their sarcasm, create avatars of my friends in the style of Disney characters, have ChatGPT summarize the DaVinci Code for my book report, and have my deep learning model suggest possible life saving drugs for me. The world is fantastic and will get better thanks to this tool, but like any tool we should use it safely and appropriately.  

Consider utilizing deep learning as an "employee" for any non-critical tasks that you don't wish to perform yourself. I personally fired my virtual assistant from Brickwork India (\$200/month) and hired ChatGPT (\$20 per month). By managing it, you can put yourself in an excellent position. In my opinion, the world won't end up in a dystopian \href{https://en.wikipedia.org/wiki/The_Terminator}{\textit{Terminator}}-esque state, nor will work disappear in a utopian \href{https://en.wikipedia.org/wiki/Fully_Automated_Luxury_Communism}{\textit{Fully-Automated Luxury Communism}} scenario. Instead, we'll find ourselves somewhere in the middle, and it'll likely be more gratifying. Work will transform, we will supervise and manage our new technological workers, and they'll be cheap! This new management job won't require us to give up our agency but to act as masters of a new realm where our attention and thought are required, and where vital decisions are still made by us.


\pagelayout{wide} % No margins
\addpart{History} % The slow drift away from algorithms
\pagelayout{margin} % Restore margins

% TODO \input{chapters/early_days.tex} % Water computers and stuff, I guess?
% TODO \setchapterpreamble[u]{\margintoc}
\chapter{Playing chess in 1997}
\labch{intro}

\section{"Textbook" AI in 1997}

Dr. Elaine Rich's textbook on Artificial Intelligence, published in the 1980s, was a groundbreaking work that helped to establish many of the foundational concepts and techniques in the field of AI. However, the rapid advancements in AI over the past few decades have led to many of the chapters in this textbook becoming obsolete.

One of the main reasons for this is the prevalence of deep learning, big data, and large-scale statistical models in modern AI. These techniques have largely replaced the symbolic, rule-based approach to AI that was emphasized in the textbook, making many of the chapters on knowledge representation and expert systems less relevant.

Additionally, the explosion of data and the availability of powerful computing resources have made it possible to apply machine learning techniques at a scale that was previously unimaginable. This has led to the development of highly effective machine learning models that can handle complex tasks such as image and speech recognition with a high degree of accuracy, making many of the chapters on simpler machine learning techniques such as decision trees\sidenote{Although mathematically, \href{https://arxiv.org/abs/2210.05189}{Neural Networks are Decision Trees}} and linear regression less relevant. \sidecite{rich_2009} \sidenote{the book is now in its third edition and unlikely to be updated as Dr. Rich as retired \href{https://www.cs.utexas.edu/~ear/}{utexas.edu}}


We'll discuss this history and a few examples from the "early days" of AI to help us understand where we are headed.

\section{Teaching computers to translate}

Noam Chomsky is a linguist and philosopher who has made significant contributions to the field of linguistics with his theory of universal grammar. Chomsky believes that all human languages share a common underlying structure, and that this structure is innate to humans. He proposes that this innate structure is the result of a "language acquisition device" present in the human brain, which allows us to learn and produce language. Chomsky also argues that the structure of language is largely independent of its content, and that the ability to produce and understand language is a fundamental aspect of human nature. His theory has been influential in the field of linguistics and has sparked much debate and research on the nature of language and its relationship to the human mind.

For English speakers or anyone who has learned English as a second language you'll have many examples of special cases, irregular verbs, bad english and former street slang that became good and proper over time. For programmers this is a nightmare, how can we codify human knowledge in a timely fashion? If we tried to write the rules of the english language in code (which many have tried to do) the rules themselves might change before we were finished writing them.

Explicitly translating languages through code is a difficult task because it requires a thorough understanding of the grammar, vocabulary, and syntax of both languages, as well as the nuances and subtleties of their respective cultures\sidenote{For programmers this is a nightmare, how can we codify human knowledge in a timely fashion? If we tried to write the rules of the english language in code (which many have tried to do) the rules themselves might change before we were finished writing them.}. Simply coding rules for how to translate words or phrases from one language to another is not sufficient, as there are often multiple valid translations for a given phrase depending on the context in which it is used.

A more effective approach to translation is to use statistical techniques that rely on a large corpus of translated data, such as Canadian laws. This type of data-driven approach involves training a machine learning model on a large dataset of translations, allowing it to learn the patterns and relationships between the languages. The model can then use this knowledge to make educated translations of new phrases or sentences, taking into account the context in which they are used.

While this approach is not perfect, it has proven to be highly effective in machine translation and can produce accurate translations even for languages that are very different from each other. The use of a large dataset of translations also allows the model to learn from the mistakes and variations present in real-world translations, further improving its accuracy.

\section{Codified human knowledge}

When we "teach" a computer to perform a task by explicitly writing down all of the rules of that task, we are really codifying human understanding.\sidenote{Programming this way makes some software development totally boring, I almost switched my major in college to math after considering what a life would look like manually writing rules for handling "edge cases" for the rest of my natural life.} When we codify human understanding we write down every rule that we know explicitly. For small tasks we can do this with 100 percent accuracy, and only minor headache on the part of the sofware developer. 

For example, let's write a boring function to tell you the number of days for a given month. 

\begin{lstlisting}[style=kaolstplain,linewidth=1.5\textwidth]
def days_in_month(year, month):
  if month in [1, 3, 5, 7, 8, 10, 12]:
    return 31
  elif month in [4, 6, 9, 11]:
    return 30
  elif month == 2:
    if (year % 4 == 0 and year % 100 != 0) or year % 400 == 0:
      return 29
    else:
      return 28
  else:
    return "Invalid month"

\end{lstlisting}

Writing code can be a tedious and repetitive task, especially when it comes to debugging and testing. It can be especially frustrating when you're working on a large project and you're trying to track down a specific bug that's causing the program to crash. Testing code can also be boring, as it often involves running the same tests over and over again to ensure that the code is working correctly.

Additionally, writing code can be boring because it requires a lot of concentration and focus. It can be easy to get lost in the details and lose track of time, especially if you're working on a complex problem. It can also be challenging to come up with creative solutions to problems, and it can be frustrating when your code doesn't work as expected.

While writing and testing code can be rewarding and fulfilling, it can also be a tedious and boring process. It requires a lot of patience, persistence, and attention to detail, and it can be easy to get frustrated and lose motivation. However, with practice and perseverance, it is possible to overcome these challenges and find enjoyment in the process of writing and testing code.

AI has traditionally operated by explicitly codifying human knowledge into machine-readable formats by doing the boring job of coding. This approach, which I'm calling "codified human knowledge" relies on humans to carefully structure and organize information in a way that can be understood by the AI system. The AI system then uses this structured knowledge to make decisions and perform tasks.

However, recent advances in AI have largely ignored the knowledge representation problem and instead have focused on using statistical techniques and neural networks to automatically learn patterns and relationships in data. This approach, known as "deep learning," involves training large neural networks on vast amounts of data, allowing the AI system to make educated classifications and transformations of data without explicit human guidance.

Deep learning has proven to be highly effective in a variety of applications, such as image and speech recognition, and has contributed to the rapid progress we have seen in AI in recent years. However, the reliance on large amounts of data and the lack of transparency in these models can make it difficult to understand how they are making decisions, which can be a concern in certain applications (hence the title of this book).

\section{Deep Blue's Brute Force Victory}

Deep Blue was a revolutionary computer developed by IBM that was specifically designed to play chess at the highest level. It was programmed with a vast database of chess knowledge and was able to analyze millions of positions per second.

Garry Kasparov was the reigning world chess champion at the time, and he was considered to be one of the greatest players in history. He had never lost a match to a computer before, and he was confident that he would be able to defeat Deep Blue.

However, things did not go as Kasparov had expected. Deep Blue was able to analyze the positions on the board with incredible speed and accuracy, and it was able to come up with highly sophisticated strategies that Kasparov had never seen before.

Despite Kasparov's best efforts, he was no match for the sheer brute force of Deep Blue's computational power. In the end, Deep Blue emerged victorious, defeating Kasparov in a historic match that changed the world of chess forever.

Deep Blue was a turning point in the development of AI, but Deep Blue's methods (namely calculating every possible outcome of a Chess game to determine the best move) was not suitable for many of the world's problems. It turns out that Chess is fun, but the world is not like chess. The "real" future of AI was being developed elsewhere, using statistics and a toy model of the brain to solve a very practical problem for banks.

\section{Meanwhile at the bank}

It was the early 1990s and Yann LeCun was a researcher at Bell Labs in New Jersey. At the time, the process of reading and processing checks was a tedious and time-consuming task that was done manually by bank employees. LeCun saw the potential for using artificial intelligence to automate this process, and he began experimenting with using convolutional neural networks (CNNs) to recognize patterns in images of checks.

At the time, CNNs were a relatively new type of neural network that had been developed in the 1980s for image recognition tasks. They were inspired by the structure of the human visual system, and were able to process images in a way that was similar to how the human brain does.

LeCun's work was groundbreaking, and he was able to achieve impressive results using CNNs to process checks. By 1993, he had developed a system that was able to read and process checks with a high degree of accuracy, significantly reducing the amount of time and effort that was required to process checks manually.

LeCun's work on using CNNs for check processing was a major milestone in the field of artificial intelligence, and it laid the foundation for the development of many other applications of CNNs in the years that followed. Today, CNNs are widely used in a variety of applications, including facial recognition, image classification, and natural language processing. \sidenote{check out Yann LeCun domonstrating a convolutional neural network in 1993 at \href{https://www.youtube.com/watch?v=FwFduRA_L6Q}{youtube.com}}

\section{Programmer intelligence, data intelligence and artificial intelligence}

I think it's useful to separate the actors in the AI problem-space into three groups. The data, the programmer and the machine learning (or AI) together they make the programs that we use every day, and for the rest of this book I'll try and separate the discussion of the smarts of each to help us better understand the world. \sidenote{I'll also repeatedly encourage you to use the words "algorithm" and "artificial intelligence" sparingly. They'll confuse your thinking, I promise.}

Programmer intelligence refers to the ability of a human programmer to design, write, and debug computer programs. This type of intelligence involves problem-solving skills, logical thinking, and the ability to learn and adapt to new programming languages and technologies. Most books don't talk about programmer intelligence and use the even more vague word "Algorithm" to describe both the programmer's output and the AI that might contibute to decision making, which can lead to misunderstanding.

Artificial intelligence is a terrible term. It generally refers to the ability of a machine or computer system to perform tasks that would normally require human intelligence, such as learning, problem-solving, and decision-making. Artificial intelligence systems can be trained to perform a wide range of tasks, from simple tasks like recognizing patterns in data to more complex tasks like understanding and generating natural language. Becuase the term is so broad I'll avoid it, and instead talk about machine learning and deep learning instead. \sidenote{In case I am confusing you, I think that Deep Blue was made with "Programmer Intelligence" or "Codified Human Knowledge" where Yann LeCun's Check OCR is "Machine Learning" with some "Data Intelligence" baked in.}

Data intelligence \sidenote{I'm making this phrase up, my point is that there is information in the data that might not be successfully extracted by a given AI (or a human-programmer).} refers to the ability to extract meaningful insights and knowledge from large datasets. This involves using statistical and analytical methods to discover patterns and trends in data, and using this information to inform business decisions or solve problems. Data intelligence requires a combination of programming skills and statistical and analytical expertise.

Overall, while all three types of intelligence are important in the field of computer science, they involve different skill sets and focus on different aspects of problem-solving and decision-making. Programmer intelligence is essential for designing and implementing computer programs, machine learning is focused on statistically mimicing human-like output in machines, and data intelligence involves using data to inform decision-making and solve problems.

By framing intelligence in this way we can chip away at the AI myths that abound and think about what is really happening. Programmers use data to make AI, there are many places where things can go awry, and many layers of misunderstanding that can get baked into AI products.


 % Deep Blue
% TODO \input{chapters/videogames.tex} % Research this!
% TODO \input{chapters/machine_translation.tex} % Understanding parts of speech is for dummies

\pagelayout{wide} % No margins
\addpart{Data} % It's all just numbers, yo!
\pagelayout{margin} % Restore margins

% TODO \input{chapters/numerical.tex} 
% TODO \input{chapters/words.tex} 
% TODO \input{chapters/sounds.tex} 
% TODO \input{chapters/images.tex} 
% TODO \input{chapters/video.tex} 
% TODO \input{chapters/combined_datasets.tex} 

\pagelayout{wide} % No margins
\addpart{Classifiers} 
\pagelayout{margin} % Restore margins

% TODO \input{chapters/recommenders.tex} 
% TODO \input{chapters/human_faces.tex} % Criminal or not, recognize my face etc... 
% TODO \input{chapters/hate_speech.tex}... 
% TODO \input{chapters/sentiment.tex}  

\pagelayout{wide} % No margins
\addpart{Transformers} % Creativity by combination
\pagelayout{margin} % Restore margins

% TODO \input{chapters/style_transfer.tex} 
% TODO \input{chapters/translation.tex} 
% TODO \input{chapters/GPTetc.tex} % GPT-3, BERT, bloom 
% TODO \input{chapters/dalle.tex} % DALL-E, cyaiyon 

\pagelayout{wide} % No margins
\addpart{Ensembles} % Creativity by combination
\pagelayout{margin} % Restore margins

% TODO \input{chapters/ensembles.tex} 

\appendix % From here onwards, chapters are numbered with letters, as is the appendix convention
\pagelayout{wide} % No margins
\addpart{Appendix}
\pagelayout{margin} % Restore margins

\setchapterstyle{lines}
\setchapterpreamble[u]{\margintoc}
\chapter{????}
\labch{appendix}

Let's say we want to build an ensemble model to analyze poetry, put a haiku into craiyon's online shit, then we categorize the resulting photo. \sidecite{Andreu2021}



% TODO \input{chapters/self-driving-cars.tex}
% TODO \input{chapters/adtech.tex}
% TODO \input{chapters/cybersecurity.tex}
% TODO \input{chapters/medicine.tex}
% TODO \input{chapters/poetry_ensemble.tex} % visualize every stanza, classify, and sum the classification.
% TODO \input{chapters/ethics.tex}
% TODO \input{chapters/mole_checking.tex} % mole checking AI as one with few qualms in practice, but issues training on white skin

%----------------------------------------------------------------------------------------

\backmatter % Denotes the end of the main document content
\setchapterstyle{plain} % Output plain chapters from this point onwards

%----------------------------------------------------------------------------------------
%	BIBLIOGRAPHY
%----------------------------------------------------------------------------------------

% The bibliography needs to be compiled with biber using your LaTeX editor, or on the command line with 'biber main' from the template directory

\defbibnote{bibnote}{Here are the references in citation order.\par\bigskip} % Prepend this text to the bibliography
\printbibliography[heading=bibintoc, title=Bibliography, prenote=bibnote] % Add the bibliography heading to the ToC, set the title of the bibliography and output the bibliography note

%----------------------------------------------------------------------------------------
%	NOMENCLATURE
%----------------------------------------------------------------------------------------

% The nomenclature needs to be compiled on the command line with 'makeindex main.nlo -s nomencl.ist -o main.nls' from the template directory

\nomenclature{$c$}{Speed of light in a vacuum inertial frame}
\nomenclature{$h$}{Planck constant}

\renewcommand{\nomname}{Notation} % Rename the default 'Nomenclature'
\renewcommand{\nompreamble}{The next list describes several symbols that will be later used within the body of the document.} % Prepend this text to the nomenclature

\printnomenclature % Output the nomenclature

%----------------------------------------------------------------------------------------
%	GLOSSARY
%----------------------------------------------------------------------------------------

% The glossary needs to be compiled on the command line with 'makeglossaries main' from the template directory

\setglossarystyle{listgroup} % Set the style of the glossary (see https://en.wikibooks.org/wiki/LaTeX/Glossary for a reference)
\printglossary[title=Special Terms, toctitle=List of Terms] % Output the glossary, 'title' is the chapter heading for the glossary, toctitle is the table of contents heading

%----------------------------------------------------------------------------------------
%	INDEX
%----------------------------------------------------------------------------------------

% The index needs to be compiled on the command line with 'makeindex main' from the template directory

\printindex % Output the index

%----------------------------------------------------------------------------------------
%	BACK COVER
%----------------------------------------------------------------------------------------

% If you have a PDF/image file that you want to use as a back cover, uncomment the following lines

%\clearpage
%\thispagestyle{empty}
%\null%
%\clearpage
%\includepdf{cover-back.pdf}

%----------------------------------------------------------------------------------------

\end{document}
