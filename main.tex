%%%%%%%%%%%%%%%%%%%%%%%%%%%%%%%%%%%%%%%%%
% kaobook
% LaTeX Class
% Version 0.9.8 (2021/08/23)
%
% This template originates from:
% https://www.LaTeXTemplates.com
%
% For the latest template development version and to make contributions:
% https://github.com/fmarotta/kaobook
%
% Authors:
% Federico Marotta (federicomarotta@mail.com)
% Based on the doctoral thesis of Ken Arroyo Ohori (https://3d.bk.tudelft.nl/ken/en)
% and on the Tufte-LaTeX class.
% Modified for LaTeX Templates by Vel (vel@latextemplates.com)
%
% License:
% LPPL (see included MANIFEST.md file)
%
%%%%%%%%%%%%%%%%%%%%%%%%%%%%%%%%%%%%%%%%%

%----------------------------------------------------------------------------------------
%	EXAMPLE AND DOCUMENTATION OF THE KAOBOOK CLASS
%----------------------------------------------------------------------------------------

\documentclass[
    letterpaper, % Page size
    fontsize=10pt, % Base font size
    twoside=false, % Use different layouts for even and odd pages (in particular, if twoside=true, the margin column will be always on the outside)
	%open=any, % If twoside=true, uncomment this to force new chapters to start on any page, not only on right (odd) pages
	%chapterentrydots=true, % Uncomment to output dots from the chapter name to the page number in the table of contents
	numbers=noenddot, % Comment to output dots after chapter numbers; the most common values for this option are: enddot, noenddot and auto (see the KOMAScript documentation for an in-depth explanation)
]{kaobook}

%----------------------------------------------------------------------------------------
%	PACKAGES AND OTHER DOCUMENT CONFIGURATIONS
%----------------------------------------------------------------------------------------

% Choose the language
\ifxetexorluatex
	\usepackage{polyglossia}
	\setmainlanguage{english}
\else
	\usepackage[english]{babel} % Load characters and hyphenation
\fi
\usepackage[english=british]{csquotes}	% English quotes

% Load packages for testing
\usepackage{blindtext} % Print text without any meaning for testing purposes
%\usepackage{showframe} % Uncomment to show boxes around the text area, margin, header and footer
%\usepackage{showlabels} % Uncomment to output the content of \label commands to the document where they are used

% Load the bibliography package
\usepackage{kaobiblio}
\addbibresource{main.bib} % Bibliography file

% Load mathematical packages for theorems and related environments
\usepackage[framed=true]{kaotheorems}

% Load the package for hyperreferences
\usepackage{kaorefs}

\graphicspath{{examples/documentation/images/}{images/}} % Paths in which to look for images

\makeindex[columns=3, title=Alphabetical Index, intoc] % Make LaTeX produce the files required to compile the index

\makeglossaries % Make LaTeX produce the files required to compile the glossary
\newglossaryentry{computer}{
	name=computer,
	description={is a programmable machine that receives input, stores and manipulates data, and provides output in a useful format}
}

% Glossary entries (used in text with e.g. \acrfull{fpsLabel} or \acrshort{fpsLabel})
\newacronym[longplural={Frames per Second}]{fpsLabel}{FPS}{Frame per Second}
\newacronym[longplural={Tables of Contents}]{tocLabel}{TOC}{Table of Contents}

 % Include the glossary definitions

\makenomenclature % Make LaTeX produce the files required to compile the nomenclature

% Reset sidenote counter at chapters
%\counterwithin*{sidenote}{chapter}

%----------------------------------------------------------------------------------------

\begin{document}

%----------------------------------------------------------------------------------------
%	BOOK INFORMATION
%----------------------------------------------------------------------------------------


\title[Untitled Project]{Untitled \\ Project}
\subtitle{Machine Learning Metaphors \\ Machine Learning, Human Understanding \\  Machine Learning and the Human Brain \\  Kicking the tires of ML models \\ Data and decisions \\ Mechanical Knowledge: Opportunites and Limits of Machine Learning (Plinko Picture) \\ Deriving Knowledge from Data: Risks and Opportunites \\ Deriving Knowledge from Data: Changing the future, while looking at the past}

\author[Brad Flaugher]{Brad Flaugher}

\date{\today}

\publishers{}

%----------------------------------------------------------------------------------------

\frontmatter % Denotes the start of the pre-document content, uses roman numerals

%----------------------------------------------------------------------------------------
%	OPENING PAGE
%----------------------------------------------------------------------------------------

%\makeatletter
%\extratitle{
%	% In the title page, the title is vspaced by 9.5\baselineskip
%	\vspace*{9\baselineskip}
%	\vspace*{\parskip}
%	\begin{center}
%		% In the title page, \huge is set after the komafont for title
%		\usekomafont{title}\huge\@title
%	\end{center}
%}
%\makeatother

%----------------------------------------------------------------------------------------
%	COPYRIGHT PAGE
%----------------------------------------------------------------------------------------

\makeatletter
\uppertitleback{\@titlehead} % Header

\lowertitleback{
	\textbf{Disclaimer}\\
	You can edit this page to suit your needs. For instance, here we have a no copyright statement, a colophon and some other information. This page is based on the corresponding page of Ken Arroyo Ohori's thesis, with minimal changes.
	
	\medskip
	
	\textbf{No copyright}\\
	\cczero\ This book is released into the public domain using the CC0 code. To the extent possible under law, I waive all copyright and related or neighbouring rights to this work.
	
	To view a copy of the CC0 code, visit: \\\url{http://creativecommons.org/publicdomain/zero/1.0/}
	
	\medskip
	
	\textbf{Colophon} \\
	This document was typeset with the help of \href{https://sourceforge.net/projects/koma-script/}{\KOMAScript} and \href{https://www.latex-project.org/}{\LaTeX} using the \href{https://github.com/fmarotta/kaobook/}{kaobook} class.
	
	The source code of this book is available at:\\\url{https://github.com/fmarotta/kaobook}
	
	(You are welcome to contribute!)
	
	\medskip
	
	\textbf{Publisher} \\
	First printed in May 2019 by \@publishers
}
\makeatother

%----------------------------------------------------------------------------------------
%	DEDICATION
%----------------------------------------------------------------------------------------

\dedication{
	The harmony of the world is made manifest in Form and Number, and the heart and soul and all the poetry of Natural Philosophy are embodied in the concept of mathematical beauty.\\
	\flushright -- D'Arcy Wentworth Thompson
}

%----------------------------------------------------------------------------------------
%	OUTPUT TITLE PAGE AND PREVIOUS
%----------------------------------------------------------------------------------------

% Note that \maketitle outputs the pages before here

\maketitle

%----------------------------------------------------------------------------------------
%	PREFACE
%----------------------------------------------------------------------------------------

\chapter*{Preface}
\addcontentsline{toc}{chapter}{Preface} % Add the preface to the table of contents as a chapter

This book is a work in progress, I hope it helps demystify the world of deep learning as I understand it.

Humans won't be able to control superintelligent AI, talk about that here\cite{Andreu2021}

Talk about Bostrom and GPAI here, and Erdi's answer to that. \cite{Erdi2019} \cite{Bostrom2014}

Talk about the alignment problem and Ethical freakouts about AI. Talk about the big 3 from 
\cite{Christian2020}
\cite{Blackman2022Jul}

Funding and startups, everybody is doing it, I'm trying to make sense of it


\begin{flushright}
	\textit{Brad Flaugher}
\end{flushright}

\index{preface}

%----------------------------------------------------------------------------------------
%	TABLE OF CONTENTS & LIST OF FIGURES/TABLES
%----------------------------------------------------------------------------------------

\begingroup % Local scope for the following commands

% Define the style for the TOC, LOF, and LOT
%\setstretch{1} % Uncomment to modify line spacing in the ToC
%\hypersetup{linkcolor=blue} % Uncomment to set the colour of links in the ToC
\setlength{\textheight}{230\hscale} % Manually adjust the height of the ToC pages

% Turn on compatibility mode for the etoc package
\etocstandarddisplaystyle % "toc display" as if etoc was not loaded
\etocstandardlines % "toc lines as if etoc was not loaded

\tableofcontents % Output the table of contents

\listoffigures % Output the list of figures

% Comment both of the following lines to have the LOF and the LOT on 
% different pages
\let\cleardoublepage\bigskip
\let\clearpage\bigskip

\listoftables % Output the list of tables

\listoflstlistings % Output the list of listings

\endgroup

%----------------------------------------------------------------------------------------
%	MAIN BODY
%----------------------------------------------------------------------------------------

\mainmatter % Denotes the start of the main document content, resets page numbering and uses arabic numbers
\setchapterstyle{kao} % Choose the default chapter heading style

\chapter*{Introduction}
\addcontentsline{toc}{chapter}{Introduction} % Add the intro to the table of contents as a chapter

Artificial Intelligence (AI) is not magic. It's here, and it's changing the world. Deep learning is one of the most exciting and popular fields of AI, but it's not the same as the good old-fashioned rules-based AI of the past. Deep learning involves training models by repeatedly showing them large datasets and allowing the models to infer the rules between input and output data.

Deep learning models are trained on data, almost like humans. However, the quality of the data is critical to the functioning of the model. For stable and well-understood environments like chess, chemistry or Newtonian physics, we can collect and generate data and deep learning can do a tremendous amount of useful work for us. In less stable environments where the rules of the day sometimes do not reflect the rules of the past, deep learning can be less helpful and even cause real harm when naively deployed. 

I'll discuss dozens of models in detail but consider any data collected that involves complex social interactions. Start with family interactions, then romantic ones and then consider that topics like advertising, trading stocks, credit scoring and even hate speech and threat detection all might have a dynamic social component to them. Models predictive power will suffer if the past does not look like the future. This is a problem for deep learning models that are trained on old data.

Also consider that as a creator of deep learning models, I can use a model to editorialize. I can train a model on data that fits my worldview instead of data that fits the world as it is. As a user or investor, how would you stop me? My stock trading model would be the first one to get me in trouble. If it was regularly monitored and managed, my model could do about as much harm as a troublesome employee could. 

But what about my model that is used to provide online dating advice, credit scores, or acceptance to students to elite universities? It might take a few years before my editorializing was found out, depending on how well it was managed and how egregious my model's outputs are.

Some of these problems are small potatoes, who cares if online dating sites don't do good science to suggest matches? What about full-self driving though? The realization of autonomous cars we're told is perpetually years away, but is it really possible with our current roads, laws and infrastructure? What happens if a model stops being updated, and it becomes the height of fashion for kids to wear shirts with street-signs on them? Do all of the cars stop working? Also, we're calling this autopilot, but don't pilots in planes need to file flight plans, speak to other planes and take instructions from a tower? Do our cars need to do that too? 

If you understand the type of AI that is being used (rules-based or deep learning), the data it was trained on, and monitor the AI in production you are on the road to success. But, what if I told you I could use the outputs from the model you just spent a billion dollars building and very easily create a new model that performs almost as well, and spend almost nothing building it. Furthermore, if you took me to court I could show my work and prove that I made the model myself. Would this change the way you invest in artificial intelligence? Would it change the way you develop and share your models?

Back to self-driving cars, Here is how that could pan out. Imagine Honda comes from behind and creates the worlds greatest fully self-driving car and it cost them only \$1 billion to make. Now imagine I buy a couple of those Hondas, maybe 100 of them for \$30,000 each, then pay 100 drivers \$100,000 per year to drive them, and equip each honda with \$70,000 of my own computers and sensors. At the end of this endeavour I could theoretically create a decent machine learning model using the "Honda's" data for 20\% of their investment. There are of course other costs I would incur, but if you come along with me and assume that the data is the most important component of the model, then I have all of the data I need on the cheap.

I'll disucss this in detail in chapter 5, but for now let me tell you I'm skeptical of full self-driving without a lot of infrastructure changes, but I'm not skeptical of the ability to create a model that can drive a car. I'm skeptical of the ability to create a model that can drive a car in a way that is safe, reliable and that will hold up in court.   

The cost of innovating is well known, and this is why patent protection exists. But deep learning models are trained on data that is free or easy to copy and in a way that produces slightly different and seemingly random inner-workings on each training run, if this is the case then I don't see many defensible positions for innovators in machine learning. Let's consider another hypothetical from pharmacology, imagine Pfizer invented the cancer curing pill, then I put that pill in a machine that came up with a significantly different formulation that achieved the same results, and when you gave it to the expert witness chemists they would have to say "the chemistry of these two pills is fundamentally different, but they both cure cancer", which would make it very hard to defend the original cancer pill's patent in court.

Because of the way they are trained, deep learning models introduce real mathematical chaos wherever they are deployed. This leads me to a few conclusions that I'll give you here in the introduction, but explain in detail in the meat of this book: 

\begin{enumerate}
\item Deep learning models can make statistically informed decisions based on the data they were previously shown, but because of their size they can produce seemingly random results. Poor data collection (or editorializing) leads to poor results. Anything using deep learning models cannot be used by itself to make critical decisions. Said differently, there must be supervision and outside control wherever deep learning is involved.
\item Any decision involving deep learning models is functionally unexplainable, and therefore likely to get someone in trouble in court. Any domain where deep learning is used to make a decision and then asked to explain in detail how that decision was made should be greeted with a shrug from the witness stand.
\item Deep learning intellectual property is indefensible, it is built in a way that is both easily copyable, and impossible to verify that it was actually copied.
\end{enumerate}

Deep learning introduces challenges for some, but opportunities for others. I am a member of the \href{https://www.fsf.org/}{Free Software Foundation}, and from my perspective deep learning models are one type of software that might inherently support the foundation's purpose. The purpose of the foundation is to promote the universal freedom to distribute and modify computer software without restriction. If deep learning models are simultaneously powerful and free, they become rocket fuel for innovation. 

This is the silver lining to the "myths" that I'd like to discuss in this book. Deep learning is messy, data science is hard, but as a tool deep learning is absolutely mindblowing. I can rank order my emails by their sarcasm, create avatars of my friends in the style of Disney characters, have ChatGPT summarize the DaVinci Code for my book report, and have my deep learning model suggest possible life saving drugs for me. The world is fantastic and will get better thanks to this tool, but like any tool we should use it safely and appropriately.  

Consider utilizing deep learning as an "employee" for any non-critical tasks that you don't wish to perform yourself. I personally fired my virtual assistant from Brickwork India (\$200/month) and hired ChatGPT (\$20 per month). By managing it, you can put yourself in an excellent position. In my opinion, the world won't end up in a dystopian \href{https://en.wikipedia.org/wiki/The_Terminator}{\textit{Terminator}}-esque state, nor will work disappear in a utopian \href{https://en.wikipedia.org/wiki/Fully_Automated_Luxury_Communism}{\textit{Fully-Automated Luxury Communism}} scenario. Instead, we'll find ourselves somewhere in the middle, and it'll likely be more gratifying. Work will transform, we will supervise and manage our new technological workers, and they'll be cheap! This new management job won't require us to give up our agency but to act as masters of a new realm where our attention and thought are required, and where vital decisions are still made by us.


\pagelayout{wide} % No margins
\addpart{Codified Human Understanding} % The slow drift away from algorithms
\pagelayout{margin} % Restore margins

% TODO \input{chapters/AIofOld.tex} % Deep Blue Brute Force, Machine Translation, Economic Models. 

\pagelayout{wide} % No margins
\addpart{How Machines Understand Data} % It's all just numbers, yo!
\pagelayout{margin} % Restore margins

% TODO \input{chapters/numerical.tex} 
% TODO \input{chapters/words.tex} 
% TODO \input{chapters/sounds.tex} 
% TODO \input{chapters/images.tex} 
% TODO \input{chapters/video.tex} 
% TODO \input{chapters/combined_datasets.tex} 

\pagelayout{wide} % No margins
\addpart{Who makes the call?}
\pagelayout{margin} % Restore margins

% TODO \input{chapters/unsupervised.tex} 
% TODO \input{chapters/supervised.tex} 
% TODO \input{chapters/reinforcement.tex} 
% TODO \input{chapters/domain_transfer.tex} % are you predicting the right thing? Are you really predicting how valuable the company is or just whether it'll be the next meme stock?
% TODO \input{chapters/ethics.tex} % Representation, "fixing the training set" Christian2020, or the Impossibility of Fairness from a model.

\pagelayout{wide} % No margins
\addpart{Classifying Data} 
\pagelayout{margin} % Restore margins

% TODO \input{chapters/recommenders.tex} 
% TODO \input{chapters/human_faces.tex} % Criminal or not, recognize my face etc... 
% TODO \input{chapters/hate_speech.tex}... 
% TODO \input{chapters/sentiment.tex}  
% TODO \input{chapters/ethics.tex} % Online Advertising, Justice, Job Applications, Creditworthiness, Getting Insurance (Weapons of Math Destruction), Civic Life, Oneil2017 ; The Default Male, Invisible Women effects snow clearing schedules and drug discovery 

\pagelayout{wide} % No margins
\addpart{Transforming Data} % Creativity by combination
\pagelayout{margin} % Restore margins

% TODO \input{chapters/style_transfer.tex} 
% TODO \input{chapters/translation.tex} 
% TODO \input{chapters/GPTetc.tex} % GPT-3, BERT, bloom 
% TODO \input{chapters/dalle.tex} % DALL-E, cyaiyon 
% TODO \input{chapters/ethics.tex}

\pagelayout{wide} % No margins
\addpart{Ensembles and Chaos} % Creativity by combination
\pagelayout{margin} % Restore margins

% TODO \input{chapters/ensembles.tex} 
% TODO \input{chapters/mathematical_chaos.tex} 

\appendix % From here onwards, chapters are numbered with letters, as is the appendix convention
\pagelayout{wide} % No margins
\addpart{Appendix}
\pagelayout{margin} % Restore margins

\setchapterstyle{lines}
\setchapterpreamble[u]{\margintoc}
\chapter{????}
\labch{appendix}

Let's say we want to build an ensemble model to analyze poetry, put a haiku into craiyon's online shit, then we categorize the resulting photo. \sidecite{Andreu2021}



% TODO \input{chapters/ethics.tex}

%----------------------------------------------------------------------------------------

\backmatter % Denotes the end of the main document content
\setchapterstyle{plain} % Output plain chapters from this point onwards

%----------------------------------------------------------------------------------------
%	BIBLIOGRAPHY
%----------------------------------------------------------------------------------------

% The bibliography needs to be compiled with biber using your LaTeX editor, or on the command line with 'biber main' from the template directory

\defbibnote{bibnote}{Here are the references in citation order.\par\bigskip} % Prepend this text to the bibliography
\printbibliography[heading=bibintoc, title=Bibliography, prenote=bibnote] % Add the bibliography heading to the ToC, set the title of the bibliography and output the bibliography note

%----------------------------------------------------------------------------------------
%	NOMENCLATURE
%----------------------------------------------------------------------------------------

% The nomenclature needs to be compiled on the command line with 'makeindex main.nlo -s nomencl.ist -o main.nls' from the template directory

\nomenclature{$c$}{Speed of light in a vacuum inertial frame}
\nomenclature{$h$}{Planck constant}

\renewcommand{\nomname}{Notation} % Rename the default 'Nomenclature'
\renewcommand{\nompreamble}{The next list describes several symbols that will be later used within the body of the document.} % Prepend this text to the nomenclature

\printnomenclature % Output the nomenclature

%----------------------------------------------------------------------------------------
%	GLOSSARY
%----------------------------------------------------------------------------------------

% The glossary needs to be compiled on the command line with 'makeglossaries main' from the template directory

\setglossarystyle{listgroup} % Set the style of the glossary (see https://en.wikibooks.org/wiki/LaTeX/Glossary for a reference)
\printglossary[title=Special Terms, toctitle=List of Terms] % Output the glossary, 'title' is the chapter heading for the glossary, toctitle is the table of contents heading

%----------------------------------------------------------------------------------------
%	INDEX
%----------------------------------------------------------------------------------------

% The index needs to be compiled on the command line with 'makeindex main' from the template directory

\printindex % Output the index

%----------------------------------------------------------------------------------------
%	BACK COVER
%----------------------------------------------------------------------------------------

% If you have a PDF/image file that you want to use as a back cover, uncomment the following lines

%\clearpage
%\thispagestyle{empty}
%\null%
%\clearpage
%\includepdf{cover-back.pdf}

%----------------------------------------------------------------------------------------

\end{document}
